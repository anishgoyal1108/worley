\documentclass{article}

% For Unicode support
\usepackage{xeCJK}

% APA Citation
\usepackage[
  style           = apa,
  citestyle       = authoryear,
  sorting         = nyt,
  sortcites       = true,
  autocite        = inline,
  citetracker     = false,
  maxbibnames     = 99,
  maxcitenames    = 2,
  backend         = biber,
  isbn            = false,
  doi             = true,
  urldate         = short,
  backend         = biber,
  defernumbers    = true
]{biblatex}

\DeclareBibliographyCategory{printcite}
\newcommand{\printcite}[1]{%
  \addtocategory{printcite}{#1}%
  \defbibcheck{key#1}{
    \iffieldequalstr{entrykey}{#1}
    {}
  {\skipentry}}%
  \printbibliography[heading=none,check=key#1]%
}
\addbibresource{cite.bib}

% Provide support on formatting SI Unit
\usepackage{siunitx}
\sisetup{per-mode=fraction}

% Math package
\usepackage{amsmath}
\renewcommand{\frac}{\dfrac}

\newcounter{source}
\newcommand{\sourcemeta}[3]{\subsection{Student Researcher} #1 %
  \subsection{Type} #2 %
\subsection{Citation} \printcite{#3}}

\newcommand{\source}[3]{\stepcounter{source} %
  \section{Source \#\thesource} %
\sourcemeta{#1}{#2}{#3}}

% Customized reflection entry
\newcounter{reflection}
\newcommand{\reflection}[2]{\stepcounter{reflection} %
  \section{Reflection \#\thereflection} 

  \paragraph{Date} #1

  % \noindent Log what you have done, what you have discovered, what you have learned, what are your next steps\ldots 
  \vspace*{-0.5cm}
\paragraph{Initials} #2}

% Customized reference
\usepackage[hidelinks]{hyperref}

% Better typesetting
\usepackage{microtype}

% Menukeys
\usepackage[os=win]{menukeys}

% Table
\usepackage{tabularx}
\usepackage{booktabs}
\usepackage{multirow}
\usepackage{makecell}
\newcolumntype{b}{>{\centering\arraybackslash}X}
\newcolumntype{s}{>{\hsize=.4\hsize\centering}X}

% Float
\usepackage{newfloat}

% 1.15 line spacing
\usepackage{setspace}
\setstretch{1.5}

% Subfigure
\usepackage{subcaption}
\usepackage{caption}

% Finer geometry
\usepackage{geometry}
\geometry{a4paper}

% Define some constants
\renewcommand{\title}{Project WORLEY}

% Define field input, i.e., box
\usepackage[most]{tcolorbox}
\newenvironment{field}{\begin{tcolorbox}[%
    enhanced, 
    breakable,
    colback = white, colframe = black,
    sharp corners,
    boxrule = 0pt, bottomrule = 1pt, toprule = 1pt,
    leftrule = 0.5pt, rightrule = 0.5pt
]{}}{\end{tcolorbox}}

% Hanging indent
\usepackage{hanging}

% Enhanced list
\usepackage{enumitem}
\setitemize{noitemsep}
\setenumerate{noitemsep}

% Reset section number within part
\usepackage{chngcntr}
\counterwithin*{section}{part}

% Footer
\usepackage{fancyhdr}
\pagestyle{fancy}
\fancyhf[FR]{\hyperlink{toc}{Return to Table of Contents}}

% Table of contents title change
\renewcommand{\contentsname}{Table of Contents}

% Part and Subpart only
\setcounter{tocdepth}{-1}

\begin{document}
\begin{titlepage}
  \centering
  \vspace*{1in}
  {\fontsize{48pt}{\baselineskip}\selectfont \bfseries
  \title}
  \vfill

  \Large
  Governor's Honors Program\\
  Statesboro, Georgia

  \vspace{0.5in}
  \setstretch{1.15}\selectfont

  \vspace{1em}
  \textbf{Team Leader}\\ Anish Goyal

  \vspace{1em}
  \textbf{Team Members}\\ Yubo Cao \& Ian Oberbeck

  \vspace{1em}
  \textbf{Teacher}\\ Kai Ouyang, Craig Worley, Goli Anupam, \& Dejah Crossfield
\end{titlepage}
\tableofcontents
\newpage

\include{directions-and-tips}
\part{Brainstorming}

\begin{itemize}
    \item A robotic hand that can act out American Sign Language (ASL) can be used for someone who relies on ASL as their primary communication language to be able to communicate independently, without relying on another person to interpret.
    \item The idea is further extended to a robotic arm that can act out ASL, which can be used to teach ASL to people who are interested in learning it.
    \item Finally, more research made us realize ASL is a language on its own, and it is not a direct translation of English. Therefore, we decided to make a robotic arm that can act out ASL, and also translate English to ASL using a BERT model.
\end{itemize}
\addtocontents{toc}{\protect\hypertarget{toc}{}}
\part{Research}

\part{Reflection}
\reflection{7/1/2023}{Anish Goyal}

For our \href{https://gosa.georgia.gov/governors-honors-program}{\gls{gosa} \gls{ghp}} final engineering project, we wanted to create an innovative device that could improve the lives of others. Our best friend, Cory, suffers from hearing loss and has to use hearing aids or sign language to communicate properly. This inspired us to create an application that signs out the \gls{asl} alphabet using a mechanical robot hand from voice input in real time.

Even though it's the first day, we have already faced skepticism from our engineering instructors regarding our ability to achieve ten \gls{dof} within a tight timeframe of two weeks. However, this skepticism has only fueled our determination to prove them wrong.

Here is some of the stuff we came up with during our brainstorming session today:
\begin{description}
\item[Pybluez] This Python library allowed us to establish a Bluetooth connection between our mobile application and the robot hand. It provided an interface to communicate and control the hand's movements seamlessly.

\item[RPI GPIO programming] To interface with the robot hand, we leveraged \gls{rpi}'s \gls{gpio} pins. By connecting our servos to pins, we will be able to control the hand's actuators and enable precise finger movements.

\item[UDP and Custom audio transmission protocol] We want to implement \gls{udp} to transmit audio data from the mobile application to the signal processing module. We could also design a custom audio transmission protocol to ensure efficient and reliable data transfer.

\item[Zero-copy memory-efficient thread-safe asynchronous queue] This concept will help us optimize the data processing pipeline by minimizing memory overhead and ensuring thread safety. By utilizing an asynchronous queue, we can achieve efficient parallel processing of audio data.

\item[Real time speech recognition through Whisper] We want to integrate cutting-edge techniques such as sliding window-based speech recognition and speaker diarization using the Whisper \gls{asr} model by OpenAI. Cutting the live audio into segments and feeding them into Whisper yields an accurate transcription of voice input with the illusion of real-time recognition.

\item[Kubeflow-powered ML operations] We want to employ Kubeflow, a \gls{ml} toolkit for Kubernetes, to build a robust \gls{ml} pipeline. With Kubeflow, we can ensure data provenance, making it possible to trace and understand the entire ML workflow.

[Ray cluster-powered hyperparameter tuning] Using Ray, an open-source framework for distributed computing to optimize our ML models, will greatly accelerate the process of hyperparameter tuning, enabling better feature recognition and extraction.
\end{description}
\newpage

\reflection{7/3/2023}{Yubo Cao}

Throughout the last Saturday \& Sunday, we have been working on a presentation for the project pitch, computer-aided design (CAD) of the model, \& the preliminary testing of the servo motion. Research on existing solutions of speech recognition, speaker diarisation (SD), voice activation detection (VAD), \& speech-to-text (STT) has also been conducted. However, After the meeting with Mr.~Kai, we have decided to focus on the implementation of robotic finger and hand, creating the backlog \& minimum viable product list (MVP), and following an incremental goal to assure the project's success in the end.

\newpage
\part{Initial Proposal Form}

\section{Project Title}

Don't worry...you don't have to commit to this title!\\
\emph{No longer than 65 characters}

\begin{field}
  \title
\end{field}

Science/engineering fair project category

\begin{field}
  Robotics and Intelligent Machines (ROBO)
\end{field}

\section{Background Research}

\subsection{Rationale For Your Project}

What similar devices/technology/research studies have investigated similar ideas?

\begin{field}

\end{field}

\subsection{Why is this research project important/needed (background research needs to support this claim)?}

\emph{Explain any societal impact of your research/include any personal
connections to the project.}

\begin{field}

\end{field}

\subsection{Major science/engineering principles and concepts}


\emph{What you will need to understand to fully understand the scope of
your project.}

\begin{field}
  
\end{field}

\subsection{Objective (science project only) - start with big picture for now}

\emph{Provide context for your investigation with a need, who would be
  interested in your investigation, what you will test and vary, and what outcomes (relationships or trends) you expect to find.}

\begin{field}

\end{field}

\section{Methodology}

\emph{Science project: general procedure/timeline of events to test hypothesis}

\subsection{Independent variable and treatment levels}

\begin{field}
 
\end{field}

\subsection{Dependent variable to measure effects of change}

\begin{field}

\end{field}


\subsection{Control group to obtain baseline/source of comparison}

\begin{field}

\end{field}


\subsection{Sample size: trials per treatment level/total}

\begin{field}
  
\end{field}

\subsection{Materials/Equipment}

\emph{List key items that need to be ordered/purchased, the vendor, and
cost}

\begin{field}
 
\end{field}

\subsection{Procedures}

\emph{Big picture for now as you will need to conduct additional
research to streamline your procedure.}

\begin{field}
  
\end{field}


\section{References: APA}

\begin{field}

\end{field}



\addtocontents{toc}{\protect\setcounter{tocdepth}{2}}
\include{initial-research}
\include{experiment-1}
\include{experiment-2}

\addtocontents{toc}{\protect\setcounter{tocdepth}{1}}
\include{data}
\include{statistical}
\include{result}
 
\nocite{*}
\printbibliography

\end{document} 
